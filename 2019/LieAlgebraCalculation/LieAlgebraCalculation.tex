\documentclass{beamer}
\usepackage{luatexja,setspace,listings}
\usepackage[mark=o]{dynkin-diagrams}
\usetheme{Luebeck}
\usecolortheme{seahorse}
\usefonttheme{structurebold,serif}
\setbeamertemplate{navigation symbols}{}
\setstretch{1.5}
\lstset{basicstyle=\ttfamily,language=Mathematica}

\title{リー代数の計算の楽しみ}
\author{宇佐見 公輔}
\date{2019年10月19日}
\begin{document}
\maketitle

\begin{frame}
    \frametitle{自己紹介}

    職業:プログラマ / 趣味:数学

    \bigskip

    関西日曜数学友の会での発表履歴:
    \begin{itemize}
        \item Generalized Onsager algebras(第5回 / 2019年8月)
        \item ルート系とディンキン図形(第4回 / 2019年4月)
        \item ラムダ計算の話(第3回 / 2018年11月)
        \item 圏論とHaskell(第2回 / 2018年8月)
    \end{itemize}

    \bigskip

    執筆参加:
    \begin{itemize}
        \item 数学デイズ大阪編:低次元のリー代数をみる \scriptsize{(Kindle 版発売中)}
    \end{itemize}
\end{frame}

\begin{frame}
    \frametitle{リー代数}

    \begin{block}{ベクトル空間とリー代数}
        ベクトル空間 = 「加法」と「スカラー倍」

        リー代数 = ベクトル空間 + 第3の演算「ブラケット積」
    \end{block}

    ブラケット積が満たすべき条件
    \begin{enumerate}
        \item \([ax + by, z] = a[x, z] + b[y, z]\), \\
              \([z, ax + by] = a[z, x] + b[z, y]\)(双線型性)
        \item \([x, x] = 0\) (\(\implies [x, y] = - [y, x]\))(交代性)
        \item \([x, [y, z]] + [y, [z, x]] + [z, [x, y]] = 0\)(Jacobi identity)
    \end{enumerate}
\end{frame}

\begin{frame}
    \frametitle{抽象的に与えられたリー代数}

    いくつかの「生成元」を用意して、加法、スカラー倍、ブラケット積をほどこして得られるものの集合を考える。

    \begin{block}{\(A_1\):生成元と関係式で与えられたリー代数}
        生成元:\(e,\, f,\, h\)

        関係式:\([e, f] = h, \quad [h, e] = 2e, \quad [h, f] = -2f\)
    \end{block}

    これはどのようなリー代数か?
\end{frame}

\begin{frame}
    \frametitle{「どのようなリー代数か」とは}

    何が分かったらいいのか?

    \begin{itemize}
        \item リー代数はベクトル空間なのだから、基底が知りたい。
        \item そして、その基底同士のブラケット積が知りたい。
    \end{itemize}

    \bigskip

    特に、行列のリー代数であらわすことができれば分かりやすい。
\end{frame}

\begin{frame}
    \frametitle{行列のリー代数}

    \begin{block}{\(\mathfrak{gl}(n,\mathbb{C})\)}
        \(\mathbb{C}\) 成分の \(n\) 次正方行列がなすベクトル空間 + 次のブラケット積
        \[
            [X, Y] := XY - YX
        \]
    \end{block}

    このブラケット積の定義は、リー代数の条件を満たしていることが確認できる。
\end{frame}

\begin{frame}
    \frametitle{抽象的なリー代数と行列のリー代数との対応}

    \begin{block}{\(\mathfrak{sl}(2,\mathbb{C})\)}
        \(\mathfrak{sl}(2,\mathbb{C}) := \{ X \in \mathfrak{gl}(2,\mathbb{C}) \mid \mathrm{tr}(X) = 0 \} \)
    \end{block}

    これは \(3\) 次元のベクトル空間で、以下の \(E,\, F,\, H\) を基底に持つ。
    \[
        E := \begin{pmatrix}
            0 & 1 \\
            0 & 0
        \end{pmatrix}, \quad
        F := \begin{pmatrix}
            0 & 0 \\
            1 & 0
        \end{pmatrix}, \quad
        H := \begin{pmatrix}
            1 & 0  \\
            0 & -1
        \end{pmatrix}
    \]
    \[
        [E, F] = H, \quad [H, E] = 2E \quad [H, F] = -2F
    \]

    \(\mathfrak{sl}(2,\mathbb{C})\) は、生成元と関係式のリー代数 \(A_1\) と同型である。
\end{frame}

\begin{frame}
    \frametitle{実際に調べたいリー代数}

    \begin{block}{\(D_n^{(1)}\) 型 Onsager 代数}
        生成元:\(e_0,\, e_1,\, \dots,\, e_n\)

        関係式:
        \begin{itemize}
            \item \([[e_i, e_j], e_j] = e_i\) (\(i\) と \(j\) が \(D_n^{(1)}\) 型 Dynkin 図形で隣り合う)
            \item \([e_i, e_j] = 0\) (otherwise)
        \end{itemize}
    \end{block}

    \scalebox{1.5}{
        \dynkin[labels={0,1,2,3,,n-2,n-1,n},edge length=1cm]{D}[1]{}
    }
\end{frame}

\begin{frame}
    \frametitle{おおまかな予想}

    \(A_n^{(1)}\) 型 Onsager 代数が、
    loop 代数 \(\mathbb{C}[t,t^{-1}] \otimes \mathfrak{sl}(n,\mathbb{C})\) の
    部分代数として具体的に実現できることは、先人の結果で分かっていた。

    \(D_n^{(1)}\) 型 Onsager 代数は、
    loop 代数 \(\mathbb{C}[t,t^{-1}] \otimes \mathfrak{o}(2n,\mathbb{C})\) の
    部分代数として書けるだろうと予想できた。

    \bigskip

    (実際、これは正しかった。\\
    参考:関西日曜数学友の会第5回 Generalized Onsager algebras)
\end{frame}

\begin{frame}
    \frametitle{どうやって調べるのか?}

    これを調べる時点で理論的な背景は分からなかったので力技。

    同型になる部分代数を探して、基底を見つけたい。
    そのために、行列同士のブラケット積をたくさん計算する必要がある。

    行列を直接計算しても良いが、以下の関係を利用してみる。

    \begin{block}{\(\mathfrak{gl}(n,\mathbb{C})\) の基底のブラケット積}
        \(E_{ij}\): (\(i,j\)) 成分だけ \(1\) で他は \(0\) の行列
        \[
            [E_{ij}, E_{kl}] = \delta_{jk}E_{il} - \delta_{il}E_{kj}
        \]
    \end{block}
\end{frame}

\begin{frame}
    \frametitle{パターンマッチ}

    ある式の中に \([E_{ij}, E_{kl}]\) という形を見つけたら、
    機械的に \(\delta_{jk}E_{il} - \delta_{il}E_{kj}\) に置き換えることができる。

    このルールだけあれば、\(E_{ij}\) が行列をあらわしたものであることは忘れてしまってもいい。

    \bigskip

    この置き換えは、プログラミングで言う「パターンマッチ」で処理できるのではないか?

    その考えに基づいてプログラムコードを書いてみる。
\end{frame}

\begin{frame}[fragile]
    \frametitle{Mathematica を使う}

    \begin{exampleblock}{Mathematica によるパターンマッチプログラム}
        \begin{lstlisting}
LieBracket[e[i_,j_],e[k_,l_]] :=
    KroneckerDelta[j,k] e[i,l]
    - KroneckerDelta[i,l] e[k,j];
LieBracket[e[1,2],e[2,3]] (* = e[1,3] *)
LieBracket[e[4,5],e[5,4]]
(* = e[4,4] - e[5,5] *)
        \end{lstlisting}
    \end{exampleblock}
\end{frame}

\begin{frame}
    \frametitle{D 型の基底}

    \begin{block}{\(\mathfrak{o}(2n,\mathbb{C})\)}
        \(\mathfrak{o}(2n,\mathbb{C}) := \{ X \in \mathfrak{gl}(2n,\mathbb{C}) \mid X^{\intercal}S + SX = 0 \} \)

        (\(S\): (\(i,j\)) 成分が \(i + j = 2n + 1\) のときだけ \(1\) で他は \(0\) の行列)
    \end{block}

    \begin{block}{\(\mathfrak{o}(2n,\mathbb{C})\) の基底のブラケット積}
        \(G_{ij} := E_{ij} - E_{2n+1-j,2n+1-i}\)
        \begin{align*}
            [G_{ij}, G_{kl}] = & \delta_{jk}G_{il} - \delta_{il}G_{kj} \\
                               & + \delta_{2n+1-j,l}G_{k,2n+1-i} - \delta_{2n+1-i,k}G_{2n+1-j,l}
        \end{align*}
    \end{block}
\end{frame}

\begin{frame}[fragile]
    \frametitle{D 型の計算}

    \begin{exampleblock}{D 型を計算するパターンマッチプログラム}
        \begin{lstlisting}
G[i_,j_] := 0 /; i+j == 2n+1;
G[i_,j_] := - G[2n+1-j,2n+1-i] /; i+j > 2n+1;
LieBracket[G[i_,j_],G[k_,l_]] :=
    KroneckerDelta[j,k] G[i,l]
    - KroneckerDelta[i,l] G[k,j]
    + KroneckerDelta[2n+1-j,l] G[k,2n+1-i]
    - KroneckerDelta[2n+1-k,i] G[2n+1-j,l];
LieBracket[G[1,2],G[2,3]] (* = G[1,3] *)
        \end{lstlisting}
    \end{exampleblock}
\end{frame}

\begin{frame}[fragile]
    \begin{exampleblock}{例:生成元の表現を探す}
        \begin{lstlisting}
e[i_]:= t[1]G[2n-1,1] + t[-1]G[1,2n-1] /;i==0;
e[i_]:= G[i,i+1] + G[i+1,i] /;1<=i<=n-1;
e[i_]:= G[n-1,n+1] + G[n+1,n-1] /;i==n;
LieBracket[e[1],e[2]]
(* = G[1,2] + G[2,1] *)
LieBracket[e[1],e[2],e[3]]
(* = G[1,3] - G[3,1] *)
LieBracket[e[1],e[2],e[3],e[4]]
(* = G[1,4] + G[4,1] *)
        \end{lstlisting}
    \end{exampleblock}
\end{frame}

\begin{frame}
    \frametitle{まとめ}

    リー代数の計算の手助けとして、Mathematica プログラミングを活用した。

    行列を計算する代わりにパターンマッチを活用することで、計算の見通しも良くなった。
    様々な組み合わせを試した結果、生成元と基底をローラン多項式+行列で表現できた。
    (Date, Usami, On an analog of the Onsager algebra of Type \(D_n^{(1)}\))

    \bigskip

    プログラミングを数学研究に活用するのも楽しい!
\end{frame}

\end{document}
