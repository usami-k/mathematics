\documentclass{jlreq}
\usepackage[pdfusetitle]{hyperref}
\hypersetup{colorlinks,allcolors=blue}
\usepackage{luatexja-fontspec}
\setmainjfont{YuKyo_Yoko-Medium}[BoldFont=YuKyo_Yoko-Bold]
\setsansjfont{YuGo-Medium}[BoldFont=YuGo-Bold]
\usepackage{stix2}
\usepackage[scaled]{helvet}
\usepackage{inconsolata}
\usepackage{mathtools}
\mathtoolsset{showonlyrefs=true}
\numberwithin{equation}{section}
\usepackage{amsthm,thmtools}
\declaretheoremstyle[thmbox={style=M,bodystyle=\upshape}]{usmstyle}
\declaretheoremstyle[qed=\qedsymbol,bodyfont=\upshape]{usmproofstyle}
\declaretheorem[style=usmstyle,numberwithin=section,name=Definition]{definition}
\declaretheorem[style=usmstyle,sibling=definition,name=Theorem]{theorem}
\declaretheorem[style=usmstyle,sibling=definition,name=Lemma]{lemma}
\declaretheorem[style=usmstyle,sibling=definition,name=Example]{example}
\declaretheorem[style=usmproofstyle,numbered=no,name=Proof]{usmproof}
\renewenvironment{proof}{\begin{usmproof}}{\end{usmproof}}
\title{ロドリゲスの回転公式}
\author{宇佐見 公輔}
\date{2022年5月5日}
\begin{document}
\maketitle

3次元の回転変換をベクトルで記述する、ロドリゲスの回転公式を紹介します。なお、この内容は先日開催された\href{https://tsudoionline.netlify.app/03/}{第3回すうがく徒のつどい}の「四元数と回転」で話した内容の一部です。その際の\href{https://github.com/usami-k/mathematics/blob/master/2022/QuaternionRotationTsudoi/QuaternionRotationTsudoi.pdf}{講演資料}には画像がありませんでしたが、今回は画像を作成しました。

\section{ロドリゲスの回転公式}

以下、3次元ベクトル空間$\mathbb{R}^3$で考えます。ベクトル$\vec{x}$と$\vec{y}$の内積を$\langle\vec{x}|\vec{y}\rangle$と、外積を$\vec{x}\times\vec{y}$と書くことにします。

3次元空間のなかで、原点を通る回転軸の周りに回転角$\theta$だけ回転するという変換は、次のように記述できます。

\begin{theorem}[ロドリゲスの回転公式]
    大きさ$1$のベクトル$\vec{n}$があるとします。点$X$を$\vec{n}$の周りに角$\theta$だけ回転した点を$X'$とします。$X$の位置ベクトルを$\vec{x}$、$X'$の位置ベクトルを$\vec{x'}$とするとき、次が成り立ちます。
    \begin{equation}
        \vec{x'}=\cos\theta\vec{x}+(1-\cos\theta)\langle\vec{n}|\vec{x}\rangle\vec{n}+\sin\theta(\vec{n}\times\vec{x})
    \end{equation}
\end{theorem}

$\vec{n}$は回転軸に沿う単位ベクトルです。$\vec{n}$の周りに角$\theta$だけ回転するという操作を図で示すと、次のようになります。

\href{https://mathlog.info/uploads/image/20220505154815.png}{回転軸の周りの回転}

ロドリゲスの回転公式は、回転後のベクトル$\vec{x'}$を、$\vec{x}$、$\vec{n}$、$\vec{n}\times\vec{x}$の3つのベクトルの和の形で記述しています。

\section{ロドリゲスの回転公式の証明}

以下、ロドリゲスの回転公式を証明します。

まず、$\vec{x}$を$\vec{n}$と平行な方向$\vec{x}_\parallel$と垂直な方向$\vec{x}_\perp$に分解します。
\begin{equation}
    \vec{x}=\vec{x}_\parallel+\vec{x}_\perp
\end{equation}
このとき、$\vec{x}_\parallel=\langle\vec{n}|\vec{x}\rangle\vec{n}$となります。

$\vec{x'}$も同様に分解します。
\begin{equation}
    \vec{x'}=\vec{x'}_\parallel+\vec{x'}_\perp
\end{equation}
このとき、点$X'$は点$X$を$\vec{n}$の周りに回転した点であることから、$\vec{x}$と$\vec{x'}$について、$\vec{n}$と平行な方向の成分は等しくなります。つまり$\vec{x'}_\parallel=\vec{x}_\parallel$です。

\href{https://mathlog.info/uploads/image/20220505161632.png}{位置ベクトルの分解}

よって、$\vec{x'}_\parallel$は$\vec{x}$と$\vec{n}$であらわせることが分かりました。次に、$\vec{x'}_\perp$を$\vec{x}$と$\vec{n}$であらわすことを考えます。

外積$\vec{n}\times\vec{x}_\perp$を考えます。外積の大きさは平行四辺形の面積でしたから、これは実は$\vec{n}\times\vec{x}$と等しいです。

$\vec{n}$と$\vec{x}_\perp$は直交していることから、次が成り立ちます。
\begin{equation}
    |\vec{n}\times\vec{x}_\perp|=|\vec{n}||\vec{x}_\perp|\sin\frac{\pi}{2}=|\vec{n}||\vec{x}_\perp|=|\vec{x}_\perp|
\end{equation}
よって、$\vec{x}_\perp$、$\vec{x'}_\perp$、$\vec{n}\times\vec{x}_\perp$はすべて同じ大きさです。

\href{https://mathlog.info/uploads/image/20220505162203.png}{外積ベクトル}

$\vec{x'}_\perp$を$\vec{x}_\perp$方向と$\vec{n}\times\vec{x}_\perp$方向に分解します。
$\vec{x}_\perp$と$\vec{x'}_\perp$のなす角は$\theta$なので、次のようになります。
\begin{equation}
    \vec{x'}_\perp=\cos\theta\vec{x}_\perp+\sin\theta(\vec{n}\times\vec{x}_\perp)
\end{equation}

\href{https://mathlog.info/uploads/image/20220505162643.png}{ベクトルの分解}

これで、$\vec{x'}_\perp$を$\vec{x}$と$\vec{n}$であらわすことができました。したがって、次が成り立ちます。
\begin{align}
    \vec{x'} & =\vec{x'}_\parallel+\vec{x'}_\perp                                                                     \\
             & =\vec{x}_\parallel+\cos\theta\vec{x}_\perp+\sin\theta(\vec{n}\times\vec{x}_\perp)                      \\
             & =\vec{x}_\parallel+\cos\theta(\vec{x}-\vec{x}_\parallel)+\sin\theta(\vec{n}\times\vec{x})              \\
             & =\cos\theta\vec{x}+(1-\cos\theta)\vec{x}_\parallel+\sin\theta(\vec{n}\times\vec{x})                    \\
             & =\cos\theta\vec{x}+(1-\cos\theta)\langle\vec{n}|\vec{x}\rangle\vec{n}+\sin\theta(\vec{n}\times\vec{x})
\end{align}

以上でロドリゲスの回転公式が証明できました。

\end{document}
