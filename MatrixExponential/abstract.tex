\documentclass{ltjsarticle}
\usepackage{amsmath,amssymb,amsfonts,amsthm,thmtools,hyperref}
\declaretheoremstyle[thmbox={style=M,bodystyle=\upshape}]{usmstyle}
\declaretheoremstyle[qed=\qedsymbol,bodyfont=\upshape]{usmproofstyle}
\declaretheorem[style=usmstyle,numberwithin=section,name=Definition]{usmdefinition}
\declaretheorem[style=usmstyle,sibling=usmdefinition,name=Theorem]{usmtheorem}
\declaretheorem[style=usmstyle,sibling=usmdefinition,name=Proposition]{usmproposition}
\declaretheorem[style=usmstyle,sibling=usmdefinition,name=Example]{usmexample}
\declaretheorem[style=usmproofstyle,numbered=no,name=Proof]{usmproof}

\title{行列の指数関数}
\author{宇佐見 公輔}
\date{第12回関西すうがく徒のつどい}
\begin{document}
\maketitle

指数関数は微積分において重要な働きをする関数ですが、行列に対しても指数関数を考えることができます。
これは、指数関数の冪級数による定義に対して、数の代わりに行列を当てはめることで得られます。

行列 \(X\) の指数関数は次のように定義されます。
\[
    e^X := E + X + \frac{1}{2!}X^2 + \frac{1}{3!}X^3 + \cdots + \frac{1}{n!}X^n + \cdots
\]

行列の指数関数では、数の場合と同様の指数法則
\[
    e^{X + Y} = e^X e^Y
\]
が成り立ちます(ただし \(X\) と \(Y\) が可換な場合)。
また、指数関数の重要な特徴である、常微分方程式
\[
    \frac{d}{dt} F(t) = A F(t)
\]
の解となる性質も、数の場合と同様に持っています。

今回の話では、上記で述べた定義や性質を詳しく説明します。
予備知識としては、線型代数の初歩と微積分の初歩の知識があれば十分です。

また、行列の指数関数の重要な応用として、リー群とリー代数の話をします。
リー群は、多様体の構造を持つ群です。このリー群の性質を調べる方法のひとつに、それに付随するリー代数を調べる方法があります。
このリー群とリー代数の対応づけの中で、実は行列の指数関数が役に立ちます。
リー群論の中で、行列の指数関数がどのように現れるかをお話しします。

\end{document}
