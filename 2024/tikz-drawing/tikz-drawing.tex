\documentclass{beamer}
\usetheme{Luebeck}
\usecolortheme{seahorse}
\usefonttheme{structurebold,serif}
\setbeamertemplate{navigation symbols}{\usebeamerfont{footline}\insertframenumber/\inserttotalframenumber}
\setlength{\parskip}{\bigskipamount}
\usepackage{luatexja-fontspec}
\setmainfont{STIX Two Text}
\setsansfont{DejaVu Sans}
\setmonofont{0xProto}[BoldFont=0xProto Italic]
\setmainjfont{YuKyo_Yoko-Medium}[BoldFont=YuKyo_Yoko-Bold]
\setsansjfont{YuGo-Medium}[BoldFont=YuGo-Bold]
\usepackage{mathtools}
\usepackage[warnings-off={mathtools-colon,mathtools-overbracket}]{unicode-math}
\unimathsetup{math-style=ISO,bold-style=ISO}
\setmathfont{STIX Two Math}
\mathtoolsset{showonlyrefs=true}
\usepackage{subcaption}
\usepackage{minted}
\usepackage{tikz}
\usetikzlibrary{arrows.meta}
\usetikzlibrary{angles,quotes}
\def\lpiece{-- ([turn]180:1) -- ([turn]-120:1) -- ([turn]-75:0.707) -- cycle}
\def\spiece{-- ([turn]180:1) -- ([turn]-150:1) -- cycle}

\title{TikZでいろいろ作図した話}
\author{宇佐見 公輔}
\date{2024年6月29日}
\begin{document}
\maketitle

\begin{frame}[fragile=singleslide]
    \frametitle{TikZとは}

    TikZは、LaTeXで図を描画するためのパッケージです。

    \tikz \draw[thick,rounded corners=8pt]
    (0,0) -- (0,2) -- (1,3.25) -- (2,2) -- (2,0)
    -- (0,2) -- (2,2) -- (0,0) -- (2,0);

    \begin{minted}{latex}
\tikz \draw[thick,rounded corners=8pt]
(0,0) -- (0,2) -- (1,3.25) -- (2,2) -- (2,0)
-- (0,2) -- (2,2) -- (0,0) -- (2,0);
    \end{minted}
\end{frame}

\begin{frame}
    \frametitle{作図例1}

    実際の技術記事で描いた図の例を挙げます。

    TikZのコード:
    \url{https://github.com/usami-k/qiita-contents/tree/main/images/2024-rust-nannou-2}

    \begin{figure}
        \begin{tikzpicture}
            \coordinate (O) at (0,0);
            \coordinate (P) at (2,2);
            \coordinate (Q) at (-3,3);
            \fill (O) circle (2pt) node[below]{$O$};
            \fill (P) circle (2pt) node[right]{$P$};
            \fill (Q) circle (2pt) node[left]{$Q$};
            \draw[thick,-{Stealth}] (O)--(P) node[midway,below right]{$\vec{OP}$};
            \draw[dashed,-{Stealth}] (O)--(Q) node[midway,below left]{$\vec{OQ}$};
            \draw[thick,-{Stealth}] (P)--(Q) node[midway,above]{$\vec{PQ}$};
            \path (P)--(Q) coordinate[pos=0.2] (P2);
            \fill (P2) circle (2pt);
            \draw[thick,-{Stealth}] (P)--(P2) node[midway,above right]{$\vec{PQ}*0.1$};
        \end{tikzpicture}
    \end{figure}
\end{frame}

\begin{frame}[fragile=singleslide]
    \frametitle{作図例1:コード(1/4)}

    \begin{figure}
        \begin{tikzpicture}
            \coordinate (O) at (0,0);
            \coordinate (P) at (2,2);
            \coordinate (Q) at (-3,3);
            \draw (O) circle (2pt);
            \draw (P) circle (2pt);
            \draw (Q) circle (2pt);
        \end{tikzpicture}
    \end{figure}

    \begin{minted}[fontsize={\fontsize{9.5pt}{11pt}}]{latex}
\begin{tikzpicture}

%%% 座標の定義
\coordinate (O) at (0,0);
\coordinate (P) at (2,2);
\coordinate (Q) at (-3,3);
    \end{minted}
\end{frame}

\begin{frame}[fragile=singleslide]
    \frametitle{作図例1:コード(2/4)}

    \begin{figure}
        \begin{tikzpicture}
            \coordinate (O) at (0,0);
            \coordinate (P) at (2,2);
            \coordinate (Q) at (-3,3);
            \fill (O) circle (2pt) node[below]{$O$};
            \fill (P) circle (2pt) node[right]{$P$};
            \fill (Q) circle (2pt) node[left]{$Q$};
        \end{tikzpicture}
    \end{figure}

    \begin{minted}[fontsize={\fontsize{9.5pt}{11pt}}]{latex}
%%% 点の描画
\fill (O) circle (2pt) node[below]{$O$};
\fill (P) circle (2pt) node[right]{$P$};
\fill (Q) circle (2pt) node[left]{$Q$};
    \end{minted}
\end{frame}

\begin{frame}[fragile=singleslide]
    \frametitle{作図例1:コード(3/4)}

    \begin{figure}
        \begin{tikzpicture}
            \coordinate (O) at (0,0);
            \coordinate (P) at (2,2);
            \coordinate (Q) at (-3,3);
            \fill (O) circle (2pt) node[below]{$O$};
            \fill (P) circle (2pt) node[right]{$P$};
            \fill (Q) circle (2pt) node[left]{$Q$};
            \draw[thick,-{Stealth}] (O)--(P) node[midway,below right]{$\vec{OP}$};
            \draw[dashed,-{Stealth}] (O)--(Q) node[midway,below left]{$\vec{OQ}$};
            \draw[thick,-{Stealth}] (P)--(Q) node[midway,above]{$\vec{PQ}$};
        \end{tikzpicture}
    \end{figure}

    \begin{minted}[fontsize={\fontsize{9.5pt}{11pt}}]{latex}
%%% ベクトルの描画
\draw[thick,-{Stealth}] (O)--(P) node[...]{$\vec{OP}$};
\draw[dashed,-{Stealth}] (O)--(Q) node[...]{$\vec{OQ}$};
\draw[thick,-{Stealth}] (P)--(Q) node[...]{$\vec{PQ}$};
    \end{minted}
\end{frame}

\begin{frame}[fragile=singleslide]
    \frametitle{作図例1:コード(4/4)}

    \begin{figure}
        \begin{tikzpicture}
            \coordinate (O) at (0,0);
            \coordinate (P) at (2,2);
            \coordinate (Q) at (-3,3);
            \fill (O) circle (2pt) node[below]{$O$};
            \fill (P) circle (2pt) node[right]{$P$};
            \fill (Q) circle (2pt) node[left]{$Q$};
            \draw[thick,-{Stealth}] (O)--(P) node[midway,below right]{$\vec{OP}$};
            \draw[dashed,-{Stealth}] (O)--(Q) node[midway,below left]{$\vec{OQ}$};
            \draw[thick,-{Stealth}] (P)--(Q) node[midway,above]{$\vec{PQ}$};
            \path (P)--(Q) coordinate[pos=0.2] (P2);
            \fill (P2) circle (2pt);
            \draw[thick,-{Stealth}] (P)--(P2) node[midway,above right]{$\vec{PQ}*0.1$};
        \end{tikzpicture}
    \end{figure}

    \begin{minted}[fontsize={\fontsize{9.5pt}{11pt}}]{latex}
%%% 内分点の描画(TikZの座標計算の機能を使う)
\path (P)--(Q) coordinate[pos=0.2] (P2);
\fill (P2) circle (2pt);
\draw[thick,...] (P)--(P2) node[...]{$\vec{PQ}*0.1$};

\end{tikzpicture}
    \end{minted}
\end{frame}

\begin{frame}[fragile=singleslide]
    \frametitle{作図例2:極座標での指定}

    \begin{figure}
        \begin{tikzpicture}
            \coordinate (O) at (0,0);
            \coordinate (P0) at (0:3);
            \coordinate (P1) at (40:3);
            \coordinate (P2) at (80:3);
            \coordinate (P3) at (120:3);
            \coordinate (P4) at (160:3);
            \fill (P0) circle (2pt);
            \fill (P1) circle (2pt);
            \fill (P2) circle (2pt);
            \fill (P3) circle (2pt);
            \draw[thick] (P0)--(P1)--(P2)--(P3)--(P4);
            \draw[dashed] (O)--(P0);
            \draw[dashed] (O)--(P1);
            \draw[dashed] (O)--(P2);
            \draw[dashed] (O)--(P3);
        \end{tikzpicture}
    \end{figure}

    \begin{minted}[fontsize={\fontsize{9.5pt}{11pt}}]{latex}
%%% 極座標での指定 (角度:半径)
\coordinate (P0) at (0:3);
\coordinate (P1) at (40:3);
\coordinate (P2) at (80:3);
\coordinate (P3) at (120:3);
\coordinate (P4) at (160:3);
    \end{minted}
\end{frame}

\begin{frame}[fragile=singleslide]
    \frametitle{作図例2:角の図示}

    \begin{figure}
        \begin{tikzpicture}
            \coordinate (O) at (0,0);
            \coordinate (P0) at (0:3);
            \coordinate (P1) at (40:3);
            \coordinate (P2) at (80:3);
            \coordinate (P3) at (120:3);
            \coordinate (P4) at (160:3);
            \fill (P0) circle (2pt);
            \fill (P1) circle (2pt);
            \fill (P2) circle (2pt);
            \fill (P3) circle (2pt);
            \draw[thick] (P0)--(P1)--(P2)--(P3)--(P4);
            \draw[dashed] (O)--(P0);
            \draw[dashed] (O)--(P1);
            \draw[dashed] (O)--(P2);
            \draw[dashed] (O)--(P3);
            \draw pic[draw,"$\frac{2\pi}{n}$",angle eccentricity=2] {angle=P0--O--P1};
            \draw pic[draw,"$\frac{2\pi}{n}$",angle eccentricity=2] {angle=P1--O--P2};
            \draw pic[draw,"$\frac{2\pi}{n}$",angle eccentricity=2] {angle=P2--O--P3};
        \end{tikzpicture}
    \end{figure}

    \begin{minted}[fontsize={\fontsize{9.5pt}{11pt}}]{latex}
%%% 角の図示
\draw pic[draw,"$\frac{2\pi}{n}$",
               angle eccentricity=2] {angle=P0--O--P1};
\draw pic[draw,"$\frac{2\pi}{n}$",
               angle eccentricity=2] {angle=P1--O--P2};
\draw pic[draw,"$\frac{2\pi}{n}$",
               angle eccentricity=2] {angle=P2--O--P3};
    \end{minted}
\end{frame}

\begin{frame}
    \frametitle{作図例3:Square-1の作図}

    前回の日曜数学会でSquare-1の話をしたときも、TikZで描いていました。

    \begin{figure}
        \centering
        \begin{subfigure}{0.28\columnwidth}
            \centering
            \begin{tikzpicture}[scale=0.8]
                \draw (-90:1) \spiece;
                \draw (-60:1) \spiece;
                \draw (-30:1) \lpiece;
                \draw ( 30:1) \lpiece;
                \draw ( 90:1) \lpiece;
                \draw (150:1) \lpiece;
                \draw (210:1) \lpiece;
            \end{tikzpicture}
        \end{subfigure}
        \begin{subfigure}{0.28\columnwidth}
            \centering
            \begin{tikzpicture}[scale=0.8]
                \draw (-90:1) \spiece;
                \draw (-60:1) \lpiece;
                \draw (  0:1) \spiece;
                \draw ( 30:1) \lpiece;
                \draw ( 90:1) \lpiece;
                \draw (150:1) \lpiece;
                \draw (210:1) \lpiece;
            \end{tikzpicture}
        \end{subfigure}
        \begin{subfigure}{0.28\columnwidth}
            \centering
            \begin{tikzpicture}[scale=0.8]
                \draw (-90:1) \spiece;
                \draw (-60:1) \lpiece;
                \draw (  0:1) \lpiece;
                \draw ( 60:1) \spiece;
                \draw ( 90:1) \lpiece;
                \draw (150:1) \lpiece;
                \draw (210:1) \lpiece;
            \end{tikzpicture}
        \end{subfigure}
    \end{figure}

    \begin{figure}
        \centering
        \begin{subfigure}{0.18\columnwidth}
            \centering
            \begin{tikzpicture}[scale=0.7]
                \draw (-90:1) \lpiece;
                \draw (-30:1) \spiece;
                \draw (  0:1) \spiece;
                \draw ( 30:1) \spiece;
                \draw ( 60:1) \spiece;
                \draw ( 90:1) \spiece;
                \draw (120:1) \spiece;
                \draw (150:1) \spiece;
                \draw (180:1) \spiece;
                \draw (210:1) \lpiece;
            \end{tikzpicture}
        \end{subfigure}
        \begin{subfigure}{0.18\columnwidth}
            \centering
            \begin{tikzpicture}[scale=0.7]
                \draw (-90:1) \lpiece;
                \draw (-30:1) \spiece;
                \draw (  0:1) \spiece;
                \draw ( 30:1) \spiece;
                \draw ( 60:1) \spiece;
                \draw ( 90:1) \spiece;
                \draw (120:1) \spiece;
                \draw (150:1) \spiece;
                \draw (180:1) \lpiece;
                \draw (240:1) \spiece;
            \end{tikzpicture}
        \end{subfigure}
        \begin{subfigure}{0.18\columnwidth}
            \centering
            \begin{tikzpicture}[scale=0.7]
                \draw (-90:1) \lpiece;
                \draw (-30:1) \spiece;
                \draw (  0:1) \spiece;
                \draw ( 30:1) \spiece;
                \draw ( 60:1) \spiece;
                \draw ( 90:1) \spiece;
                \draw (120:1) \spiece;
                \draw (150:1) \lpiece;
                \draw (210:1) \spiece;
                \draw (240:1) \spiece;
            \end{tikzpicture}
        \end{subfigure}
        \begin{subfigure}{0.18\columnwidth}
            \centering
            \begin{tikzpicture}[scale=0.7]
                \draw (-90:1) \lpiece;
                \draw (-30:1) \spiece;
                \draw (  0:1) \spiece;
                \draw ( 30:1) \spiece;
                \draw ( 60:1) \spiece;
                \draw ( 90:1) \spiece;
                \draw (120:1) \lpiece;
                \draw (180:1) \spiece;
                \draw (210:1) \spiece;
                \draw (240:1) \spiece;
            \end{tikzpicture}
        \end{subfigure}
        \begin{subfigure}{0.18\columnwidth}
            \centering
            \begin{tikzpicture}[scale=0.7]
                \draw (-90:1) \lpiece;
                \draw (-30:1) \spiece;
                \draw (  0:1) \spiece;
                \draw ( 30:1) \spiece;
                \draw ( 60:1) \spiece;
                \draw ( 90:1) \lpiece;
                \draw (150:1) \spiece;
                \draw (180:1) \spiece;
                \draw (210:1) \spiece;
                \draw (240:1) \spiece;
            \end{tikzpicture}
        \end{subfigure}
    \end{figure}
\end{frame}

\begin{frame}[fragile=singleslide]
    \frametitle{作図例3:小ピースと大ピース}

    \begin{figure}
        \begin{subfigure}{0.28\columnwidth}
            \centering
            \caption{小ピース}
            \begin{tikzpicture}[scale=2.0]
                \draw (0:1) \spiece;
            \end{tikzpicture}
        \end{subfigure}
        \begin{subfigure}{0.28\columnwidth}
            \centering
            \caption{大ピース}
            \begin{tikzpicture}[scale=2.0]
                \draw (0:1) \lpiece;
            \end{tikzpicture}
        \end{subfigure}
    \end{figure}

    \begin{minted}[fontsize={\fontsize{9.5pt}{11pt}}]{latex}
%%% 小ピース
\draw (0:1) -- ([turn]180:1)
            -- ([turn]-150:1) -- cycle;

%%% 大ピース
\draw (0:1) -- ([turn]180:1) -- ([turn]-120:1)
            -- ([turn]-75:0.707) -- cycle;
    \end{minted}
\end{frame}

\begin{frame}[fragile=singleslide]
    \frametitle{作図例3:相対座標による小ピースの描画}

    \begin{figure}
        \begin{subfigure}{0.28\columnwidth}
            \centering
            \caption{小ピース}
            \begin{tikzpicture}[scale=2.0]
                \draw[thick,-{Stealth}] (0:1) -- ([turn]180:1) -- ([turn]-150:1);
                \draw[dashed] (0:1) \spiece;
            \end{tikzpicture}
        \end{subfigure}
        \begin{subfigure}{0.28\columnwidth}
            \centering
            \caption{始点の変更}
            \begin{tikzpicture}[scale=2.0]
                \draw[thick,-{Stealth}] (120:1) -- ([turn]180:1) -- ([turn]-150:1);
                \draw[dashed] (120:1) \spiece;
            \end{tikzpicture}
        \end{subfigure}
    \end{figure}

    \begin{minted}[fontsize={\fontsize{9.5pt}{11pt}}]{latex}
%%% 小ピース
\draw (0:1) -- ([turn]180:1)
            -- ([turn]-150:1) -- cycle;

%%% 始点の変更
\draw (120:1) -- ([turn]180:1)
              -- ([turn]-150:1) -- cycle;
    \end{minted}
\end{frame}

\begin{frame}[fragile=singleslide]
    \frametitle{作図例3:相対座標による大ピースの描画}

    \begin{figure}
        \begin{subfigure}{0.28\columnwidth}
            \centering
            \caption{大ピース}
            \begin{tikzpicture}[scale=2.0]
                \draw[thick,-{Stealth}] (0:1) -- ([turn]180:1) -- ([turn]-120:1) -- ([turn]-75:0.707);
                \draw[dashed] (0:1) \lpiece;
            \end{tikzpicture}
        \end{subfigure}
        \begin{subfigure}{0.28\columnwidth}
            \centering
            \caption{始点の変更}
            \begin{tikzpicture}[scale=2.0]
                \draw[thick,-{Stealth}] (120:1) -- ([turn]180:1) -- ([turn]-120:1) -- ([turn]-75:0.707);
                \draw[dashed] (120:1) \lpiece;
            \end{tikzpicture}
        \end{subfigure}
    \end{figure}

    \begin{minted}[fontsize={\fontsize{9.5pt}{11pt}}]{latex}
%%% 大ピース
\draw (0:1) -- ([turn]180:1) -- ([turn]-120:1)
            -- ([turn]-75:0.707) -- cycle;

%%% 始点の変更
\draw (120:1) -- ([turn]180:1) -- ([turn]-120:1)
              -- ([turn]-75:0.707) -- cycle;
    \end{minted}
\end{frame}

\begin{frame}[fragile=singleslide]
    \frametitle{作図例3:マクロを活用してピースを並べる}

    \begin{figure}
        \begin{tikzpicture}[scale=0.8]
            \draw (  0:1) \spiece;
            \draw ( 30:1) \spiece;
            \draw ( 60:1) \lpiece;
            \draw (120:1) \spiece;
            \draw (150:1) \spiece;
            \draw (180:1) \lpiece;
            \draw (240:1) \spiece;
        \end{tikzpicture}
    \end{figure}

    \begin{minted}[fontsize={\fontsize{9.5pt}{11pt}}]{latex}
%%% マクロ定義
\def\spiece{-- ([turn]180:1) -- ([turn]-150:1) -- cycle}
\def\lpiece{-- ([turn]180:1) -- ([turn]-120:1)
            -- ([turn]-75:0.707) -- cycle}
%%% ピースを並べる
\draw (  0:1) \spiece;
\draw ( 30:1) \spiece;
\draw ( 60:1) \lpiece;
\draw (120:1) \spiece;
\draw (150:1) \spiece;
\draw (180:1) \lpiece;
\draw (240:1) \spiece;
    \end{minted}
\end{frame}

\begin{frame}
    \frametitle{作図のためにLaTeX+TikZを使う}

    TikZは普通、LaTeX文書上で図を描くために使います。

    しかし使い方を覚えると、LaTeX以外でも活用したくなります。

    そんなときも、次のようにすれば対応できます。
    \begin{enumerate}
        \item LaTeX+TikZで図を描く
        \item 描いた図をPDFにする(\mintinline{latex}{preview}パッケージが便利)
        \item PDFを画像に変換する(各種変換ツールを利用)
    \end{enumerate}

    この方法で、TikZを作図ツールとして使って技術記事などで使いました。
\end{frame}

\begin{frame}
    \frametitle{まとめ}

    TikZのテクニック:

    \begin{itemize}
        \item 座標計算(内分点)
        \item 極座標での指定
        \item 角の図示
        \item 相対座標での指定
        \item マクロの活用
    \end{itemize}

    ほかにもいろいろな機能があります(TikZのマニュアルはなかなかに膨大な量です)。
\end{frame}

\end{document}
