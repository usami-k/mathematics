\documentclass{beamer}
\usetheme{Luebeck}
\usecolortheme{seahorse}
\usefonttheme{structurebold,serif}
\setbeamertemplate{navigation symbols}{\usebeamerfont{footline}\insertframenumber/\inserttotalframenumber}
\usepackage{luatexja-fontspec}
\setmainfont{DejaVu Serif}[Scale=0.9]
\setsansfont{DejaVu Sans}[Scale=0.9]
\setmonofont{DejaVu Sans Mono}[Scale=0.9]
\setmainjfont{YuKyo_Yoko-Medium}[BoldFont=YuKyo_Yoko-Bold]
\setsansjfont{YuGo-Medium}[BoldFont=YuGo-Bold]
\usepackage{hyperref}
\usepackage[bold-style=ISO]{unicode-math}
\usepackage{ulem}

\newcommand{\ii}{\mathrm{i}}
\newcommand{\jj}{\mathrm{j}}
\newcommand{\kk}{\mathrm{k}}
\newcommand{\ee}{\mathrm{e}}

\title{八元数のはなし}
\subtitle{何を「数」と呼ぶのか?}
\author{宇佐見 公輔}
\date{2021年10月23日}
\begin{document}
\maketitle

\begin{frame}
    \frametitle{数の拡張}

    \begin{block}{普通の数の拡張}
        \begin{itemize}
            \item 自然数 \(\mathbb{N}\)
            \item 整数 \(\mathbb{Z}\)
            \item 有理数 \(\mathbb{Q}\)
            \item 実数 \(\mathbb{R}\)
            \item 複素数 \(\mathbb{C}\)
        \end{itemize}
    \end{block}

    \begin{block}{さらなる数の拡張}
        \begin{itemize}
            \item 四元数 \(\mathbb{H}\)
            \item 八元数 \(\mathbb{O}\)
        \end{itemize}
    \end{block}
\end{frame}

\begin{frame}
    \frametitle{複素数と四元数}

    \begin{block}{複素数}
        \(a_0 + a_1 \ii\)
        とあらわされる数(\(a_i \in \mathbb{R}\))。
        \[
            \ii^2 = -1
        \]
    \end{block}

    \begin{block}{四元数}
        \(a_0 + a_1 \ii + a_2 \jj + a_3 \kk\)
        とあらわされる数(\(a_i \in \mathbb{R}\))。
        \begin{gather*}
            \ii^2 = \jj^2 = \kk^2 = -1 \\
            \ii \jj = -\jj \ii = \kk, \quad \jj \kk = -\kk \jj = \ii, \quad \kk \ii = -\ii \kk = \jj
        \end{gather*}
    \end{block}
\end{frame}

\begin{frame}
    \frametitle{八元数}

    \begin{block}{八元数}
        \(a_0 + a_1 \ee_1 + a_2 \ee_2 + a_3 \ee_3 + a_4 \ee_4 + a_5 \ee_5 + a_6 \ee_6 + a_7 \ee_7\) \\
        とあらわされる数(\(a_i \in \mathbb{R}\))。
        \begin{gather*}
            \ee_1^2 = \ee_2^2 = \ee_3^2 = \ee_4^2 = \ee_5^2 = \ee_6^2 = \ee_7^2 = -1 \\
            \ee_1 \ee_2 = -\ee_2 \ee_1 = \ee_3, \quad \ee_2 \ee_3 = -\ee_3 \ee_2 = \ee_1, \quad \ee_3 \ee_1 = -\ee_1 \ee_3 = \ee_2 \\
            \ee_1 \ee_4 = -\ee_4 \ee_1 = \ee_5, \quad \ee_4 \ee_5 = -\ee_5 \ee_4 = \ee_1, \quad \ee_5 \ee_1 = -\ee_1 \ee_5 = \ee_4 \\
            \ee_2 \ee_4 = -\ee_4 \ee_2 = \ee_6, \quad \ee_4 \ee_6 = -\ee_6 \ee_4 = \ee_2, \quad \ee_6 \ee_2 = -\ee_2 \ee_6 = \ee_4 \\
            \ee_3 \ee_4 = -\ee_4 \ee_3 = \ee_7, \quad \ee_4 \ee_7 = -\ee_7 \ee_4 = \ee_3, \quad \ee_7 \ee_3 = -\ee_3 \ee_7 = \ee_4 \\
            \ee_5 \ee_3 = -\ee_3 \ee_5 = \ee_6, \quad \ee_3 \ee_6 = -\ee_6 \ee_3 = \ee_5, \quad \ee_6 \ee_5 = -\ee_5 \ee_6 = \ee_3 \\
            \ee_6 \ee_1 = -\ee_1 \ee_6 = \ee_7, \quad \ee_1 \ee_7 = -\ee_7 \ee_1 = \ee_6, \quad \ee_7 \ee_6 = -\ee_6 \ee_7 = \ee_1 \\
            \ee_7 \ee_2 = -\ee_2 \ee_7 = \ee_5, \quad \ee_2 \ee_5 = -\ee_5 \ee_2 = \ee_7, \quad \ee_5 \ee_7 = -\ee_7 \ee_5 = \ee_2
        \end{gather*}
    \end{block}
\end{frame}

\begin{frame}
    \frametitle{結合法則の崩れ}

    \begin{block}{交換法則の崩れ}
        四元数と八元数は交換法則が成り立たない。
        \begin{gather*}
            \ee_1 \ee_2 \neq \ee_2 \ee_1 \\
            \ee_1 \ee_2 = \ee_3, \qquad \ee_2 \ee_1 = - \ee_3
        \end{gather*}
    \end{block}

    \begin{block}{結合法則の崩れ}
        八元数は結合法則が成り立たない。
        \begin{gather*}
            (\ee_1 \ee_2) \ee_4 \neq \ee_1 (\ee_2 \ee_4) \\
            (\ee_1 \ee_2) \ee_4 = \ee_3 \ee_4 = \ee_7, \qquad \ee_1 (\ee_2 \ee_4) = \ee_1 \ee_6 = - \ee_7
        \end{gather*}
    \end{block}
\end{frame}

\begin{frame}
    \frametitle{余談:「結合的」という言葉}

    グレイブスが八元数を発見してハミルトンに伝えたとき、ハミルトンが結合法則が成り立たないことを指摘した。
    このとき初めて「結合的(associative)」という言葉が使われた。

    \bigskip

    なお、結合的でない代数構造は、八元数のほかに、リー代数やジョルダン代数などがある。
\end{frame}

\begin{frame}
    \frametitle{何を「数」と呼ぶのか?}

    \begin{block}{疑問1}
        四元数や八元数のように、交換法則や結合法則が成り立たないものを「数」と呼んでいいのか?
    \end{block}

    \begin{block}{疑問2}
        四元数や八元数を「数」と認めるとして、それ以外のものは「数」とは呼ばないのか?
    \end{block}
\end{frame}

\begin{frame}
    \frametitle{数っぽさ}

    有理数や実数が持っている「数っぽさ」は何か?

    \begin{itemize}
        \item ものの量をあらわす
              \begin{itemize}
                  \item 大きさがある
                  \item 大小比較ができる
              \end{itemize}
        \item 加減乗除ができる
              \begin{itemize}
                  \item 加法で閉じる、結合法則、交換法則
                  \item 減法で閉じる(加法の逆元がある)
                  \item 乗法で閉じる、結合法則、交換法則、分配法則
                  \item 除法で閉じる(乗法の逆元がある)
              \end{itemize}
    \end{itemize}
\end{frame}

\begin{frame}
    \frametitle{複素数はどうか?}

    先ほどの「数っぽさ」を複素数は持っているか?

    \bigskip

    複素数どうしの大小比較はできない。
    ただし、絶対値は定義できる。\(a = a_0 + a_1 \ii\) に対して、
    \[
        |a| := \sqrt{a_0^2 + a_1^2}
    \]

    \bigskip

    加減乗除は問題ない。特に、乗法の逆元は次のようになる。
    \[
        a \overline{a} = a_0^2 + a_1^2 = |a|^2
    \]
    より、\(a \neq 0\) のとき
    \[
        a^{-1} = \frac{\overline{a}}{|a|^2}
    \]
\end{frame}

\begin{frame}
    \frametitle{絶対値に期待する性質}

    複素数は、実数のような「大小比較ができる」という性質は持たなくなった。
    この点はあきらめるが、「絶対値」はまだ持っている。

    この「絶対値」と「加減乗除」との関連を考えてみる。

    \bigskip

    絶対値は「原点からの距離」にあたるものだから、距離として次の性質は持っていてほしい。
    \[
        |a + b| \leq |a| + |b|
    \]

    また乗法は、大きさに関しては「拡大縮小」であってほしいから、次の性質は持っていてほしい。
    \[
        |ab| = |a||b|
    \]

    複素数の絶対値はこれらを満たしている。
\end{frame}

\begin{frame}
    \frametitle{四元数や八元数はどうか?}

    四元数も絶対値は定義できる。\(a = a_0 + a_1 \ii + a_2 \jj + a_3 \kk\) に対して、
    \[
        |a| := \sqrt{a_0^2 + a_1^2 + a_2^2 + a_3^2}
    \]

    八元数も同様に定義できる。
    \[
        |a| := \sqrt{a_0^2 + a_1^2 + a_2^2 + a_3^2 + a_4^2 + a_5^2 + a_6^2 + a_7^2}
    \]

    これらは、\(|a + b| \leq |a| + |b|\) や \(|ab| = |a||b|\) を満たす。

    \bigskip

    加減乗除は問題ない。乗法の逆元は、\(a \neq 0\) のとき
    \[
        a^{-1} = \frac{\overline{a}}{|a|^2}
    \]
\end{frame}

\begin{frame}
    \frametitle{数の性質}

    「数」に期待する性質は以下で、実数、複素数、四元数、八元数はこれらを満たす。

    \begin{itemize}
        \item 絶対値がある
              \begin{itemize}
                  \item \(|a + b| \leq |a| + |b|\)
                  \item \(|ab| = |a||b|\)
              \end{itemize}
        \item 加減乗除ができる
              \begin{itemize}
                  \item 加法で閉じる、結合法則、交換法則
                  \item 減法で閉じる(加法の逆元がある)
                  \item 乗法で閉じる、分配法則
                  \item 除法で閉じる(乗法の逆元がある)
              \end{itemize}
    \end{itemize}

    なお、代数の言葉を使えば、(実数体上の)「ノルム付き可除代数(normed division algebra)」である
    (乗法的なノルムを持ち、零元以外が乗法の逆元を持つ \(\mathbb{R}\) 代数)。
\end{frame}

\begin{frame}
    \frametitle{これ以外の「数」はないのか?}

    \begin{block}{定理}
        実数体上のノルム付き可除代数は、\(\mathbb{R}\)、\(\mathbb{C}\)、\(\mathbb{H}\)、\(\mathbb{O}\) の4種類しかない。
    \end{block}

    八元数を拡張して十六元数を構成することはできる。
    しかし、十六元数は以下の性質を持ち、「数」ではなくなる。

    \begin{itemize}
        \item \(|ab| = |a||b|\) とは限らない。
        \item 乗法の逆元が存在するとは限らない。特に、零因子が存在する。
    \end{itemize}

    零因子とは、\(a \neq 0\), \(b \neq 0\), \(ab = 0\) を満たす \(a\), \(b\) のこと。
    零因子は乗法の逆元を持たない。

\end{frame}

\begin{frame}
    \frametitle{参考文献}

    他にもありますが、読みやすい本として。

    \begin{block}{参考文献}
        松岡 学 \\
        「数の世界 自然数から実数、複素数、そして四元数へ」 \\
        講談社ブルーバックス
    \end{block}
\end{frame}

\end{document}
