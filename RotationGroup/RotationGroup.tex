\documentclass{beamer}
\usepackage{luatexja,setspace}
\usetheme{Luebeck}
\usecolortheme{seahorse}
\usefonttheme{structurebold,serif}
\setbeamertemplate{navigation symbols}{\usebeamerfont{footline}\insertframenumber/\inserttotalframenumber}
\setstretch{1.5}
\theoremstyle{definition}
\newtheorem{proposition}{Proposition}

\title{回転群のはなし}
\author{宇佐見 公輔}
\date{第6回 関西日曜数学 友の会}
\begin{document}
\maketitle

\begin{frame}
    \frametitle{最近の趣味数学}

    関西日曜数学 友の会:
    \begin{itemize}
        \item Generalized Onsager algebras(第5回 / 2019年8月)
        \item ルート系とディンキン図形(第4回 / 2019年4月)
    \end{itemize}

    日曜数学会:
    \begin{itemize}
        \item リー代数の計算の楽しみ(マスパーティ / 2019年10月)
    \end{itemize}

    関西すうがく徒のつどい:
    \begin{itemize}
        \item 行列の指数関数(第12回 / 2019年10月)
    \end{itemize}

    執筆参加:
    \begin{itemize}
        \item 数学デイズ大阪編:低次元のリー代数をみる \scriptsize{(Kindle 版発売中)}
    \end{itemize}
\end{frame}

\begin{frame}
    \frametitle{2次元回転行列}

    2次元平面 \(\mathbb{R}^2\) を考えます。

    ある点を、原点を中心として反時計回りに角度 \(t\) だけ回転させる作用は、
    次の行列であらわされます。

    \begin{definition}
        \[
            R(t) :=
            \begin{pmatrix}
                \cos t & - \sin t \\
                \sin t & \cos t
            \end{pmatrix}
        \]
    \end{definition}
\end{frame}

\begin{frame}
    \frametitle{2次元回転行列の積}

    \begin{proposition}
        2次元回転行列について次が成り立ちます。
        \[
            R(t_1) R(t_2) = R(t_1 + t_2)
        \]
    \end{proposition}

    これは計算すれば確認できます。

\end{frame}

\begin{frame}
    \frametitle{2次元の回転群}

    \(R(t_1) R(t_2) = R(t_1 + t_2)\) という関係から、
    \(\mathit{SO}(2) := \{R(t) \mid t \in \mathbb{R}\} \) が可換群であることが分かります。

    \begin{proposition}
        \begin{itemize}
            \item 積で閉じている:\(R(t_1) R(t_2) \in \mathit{SO}(2)\)
            \item 結合法則:\(R(t_1) (R(t_2) R(t_3)) = (R(t_1) R(t_2)) R(t_3)\)
            \item 交換法則:\(R(t_1) R(t_2) = R(t_2) R(t_1)\)
            \item 単位元:\(R(0)\) は単位元
            \item 逆元:\(R(t)\) の逆元は \(R(- t)\)
        \end{itemize}
    \end{proposition}
\end{frame}

\begin{frame}
    \frametitle{無限小の回転}

    回転角 \(t\) を「無限小」にとることを考えます。つまり、
    \begin{align*}
        \cos t & = 1 - \frac{1}{2!}t^2 + \frac{1}{4!}t^4 - \cdots \\
        \sin t & = t - \frac{1}{3!}t^3 + \frac{1}{5!}t^5 - \cdots
    \end{align*}
    のうち、2次以上の項を無視することを考えます。すると、
    \[
        R(t) =
        \begin{pmatrix}
            1 & - t \\
            t & 1
        \end{pmatrix}
        =
        \begin{pmatrix}
            1 & 0 \\
            0 & 1
        \end{pmatrix}
        + t
        \begin{pmatrix}
            0 & - 1 \\
            1 & 0
        \end{pmatrix}
    \]
    となります。
\end{frame}

\begin{frame}
    \frametitle{無限小回転の生成行列}

    先ほどの観察から、次の行列が重要そうに見えてきます。
    \[
        J :=
        \begin{pmatrix}
            0 & - 1 \\
            1 & 0
        \end{pmatrix}
    \]
    これを使って、\(R(t)\) は以下のように書けます。
    \begin{align*}
        R(t) & =
        \begin{pmatrix}
            1 & 0 \\
            0 & 1
        \end{pmatrix}
        + t
        \begin{pmatrix}
            0 & - 1 \\
            1 & 0
        \end{pmatrix} + O(t^2) \\
             & = I + t J + O(t^2)
    \end{align*}
\end{frame}

\begin{frame}
    \frametitle{行列の指数関数}

    \begin{definition}
        行列 \(X\) の指数関数を次のように定義します。
        \[
            \exp X := \sum_{k=0}^{\infty} \frac{1}{k!} X^k
            = I + X + \frac{1}{2!}X^2 + \frac{1}{3!}X^3 + \cdots + \frac{1}{k!}X^k + \cdots
        \]
    \end{definition}

    (これについては、第12回 関西すうがく徒のつどいで話しました)
\end{frame}

\begin{frame}
    \frametitle{回転行列と指数関数}

    \(R(t) = I + t J + O(t^2)\) と述べましたが、
    実は指数関数を使って次のように書けます。

    \begin{proposition}
        回転行列 \(R(t)\) は次のように書けます。
        \begin{align*}
            R(t) & = \exp (tJ)                                                                                     \\
                 & = I + tJ + \frac{1}{2!}{(tJ)}^2 + \frac{1}{3!}{(tJ)}^3 + \cdots + \frac{1}{k!}{(tJ)}^k + \cdots
        \end{align*}
    \end{proposition}
\end{frame}

\begin{frame}
    \frametitle{指数関数と三角関数}

    回転行列は \(R(t) = (\cos t) I + (\sin t) J\) とも書けるので、
    以下が分かります。

    \begin{proposition}
        次が成り立ちます。
        \[
            \exp (tJ) = (\cos t) I + (\sin t) J
        \]
    \end{proposition}
\end{frame}

\begin{frame}
    \frametitle{3次元の回転}

    3次元空間 \(\mathbb{R}^3\) での回転はもう少し複雑になります。

    2次元の場合は、原点を通る回転軸(回転面に対して垂直な直線)がひとつだけでした。
    2次元の回転は回転角という1パラメータであらわせました。

    3次元の場合は、原点を通る回転軸がひとつではありません。
    回転軸の向きを決めるためにパラメータを2つ使うため、
    回転角と合わせて3つのパラメータが必要になります。
\end{frame}

\begin{frame}
    \frametitle{3次元回転行列}

    \begin{definition}
        第1軸、第2軸、第3軸のまわりの回転行列
        \begin{align*}
            R_1(t) & :=
            \begin{pmatrix}
                1 & 0 & 0 \\
                0 & \cos t & - \sin t \\
                0 & \sin t & \cos t
            \end{pmatrix} &
            R_2(t) & :=
            \begin{pmatrix}
                \cos t & 0 & \sin t \\
                0 & 1 & 0 \\
                - \sin t & 0 & \cos t
            \end{pmatrix} \\
            R_3(t) & :=
            \begin{pmatrix}
                \cos t & - \sin t & 0 \\
                \sin t & \cos t & 0 \\
                0 & 0 & 1
            \end{pmatrix}
        \end{align*}
    \end{definition}
\end{frame}

\begin{frame}
    \frametitle{3次元の回転の行列表示}

    3次元の回転をひとつの行列で具体的に書こうとすると、少しややこしい式になります。

    しかし、3次元の回転は \(R_1(t), R_2(t), R_3(t)\) の積であらわすことができます。

    そのため、この3つの回転行列をおさえることで3次元の回転群の本質を知ることができます。
\end{frame}

\begin{frame}
    \frametitle{再び無限小の回転}

    回転角 \(t\) の「無限小」を考えます(\(t\) の2次以上を無視)。
    \begin{align*}
        R_3(t) =
        \begin{pmatrix}
            1 & - t & 0 \\
            t & 1 & 0 \\
            0 & 0 & 1
        \end{pmatrix}
        & =
        \begin{pmatrix}
            1 & 0 & 0 \\
            0 & 1 & 0 \\
            0 & 0 & 1
        \end{pmatrix}
        + t
        \begin{pmatrix}
            0 & - 1 & 0 \\
            1 & 0 & 0 \\
            0 & 0 & 0
        \end{pmatrix} \\
        & = I + t J_3
    \end{align*}

    (\(J_3\) をそのように定義する)
\end{frame}

\begin{frame}
    \frametitle{再び回転行列と指数関数}

    \begin{proposition}
        回転行列 \(R_1(t), R_2(t), R_3(t)\) は次のように書けます。
        \[
            R_1(t) = \exp (tJ_1), \quad
            R_2(t) = \exp (tJ_2), \quad
            R_3(t) = \exp (tJ_3)
        \]
        \begin{footnotesize}
            ここで
            \[
                J_1 :=
                \begin{pmatrix}
                    0 & 0 & 0 \\
                    0 & 0 & -1 \\
                    0 & 1 & 0
                \end{pmatrix}, \quad
                J_2 :=
                \begin{pmatrix}
                    0 & 0 & 1 \\
                    0 & 0 & 0 \\
                    -1 & 0 & 0
                \end{pmatrix}, \quad
                J_3 :=
                \begin{pmatrix}
                    0 & -1 & 0 \\
                    1 & 0 & 0 \\
                    0 & 0 & 0
                \end{pmatrix}
            \]
        \end{footnotesize}
    \end{proposition}
\end{frame}

\begin{frame}
    \frametitle{3次元回転の生成行列の関係}

    \begin{proposition}
        \(J_1, J_2, J_3\) の間には次の関係があります。
        \begin{align*}
            [J_1, J_2] & = - J_3 \\
            [J_2, J_3] & = - J_1 \\
            [J_3, J_1] & = - J_2
        \end{align*}

        (ここで \([X, Y] := XY - YX\))
    \end{proposition}
\end{frame}

\begin{frame}
    \frametitle{さらなる話題}

    \begin{itemize}
        \item 回転行列は、簡単な形の行列から指数関数で生成される
        \item 生成行列には、交代子積によってリー代数の構造がある
        \item そのリー代数を調べることで回転群のことがわかる
    \end{itemize}
\end{frame}

\end{document}
